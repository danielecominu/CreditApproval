% !TEX encoding = UTF-8 Unicode
% !TeX spellcheck = it
\documentclass[a4paper,8pt]{article}
\usepackage[italian]{babel}
\usepackage{graphicx}
\usepackage{blindtext}
\usepackage[T1]{fontenc}
\usepackage[utf8x]{inputenc}
\usepackage[none]{hyphenat}
\usepackage[margin={1cm,2cm}]{geometry}
\usepackage{hyperref}
\usepackage{titlesec}

\begin{document}
\title{Credit Card Approval}
\author{Valerio Colitta, Daniele Cominu, Alessio Fiorenza}
\date{Aprile 2017}
\maketitle

\begin{itemize}  
\item Classe di problema affrontato: Supervised, classificazione Binaria\\
\item Origine del dataset: \cite{Dataset}
\item Descrizione del vostro vettore delle feature x : \cite{Features} (alcuni valori per alcune xi sono mancanti. Pensavo di riempirle, facendo una media delle altre xi)
\item Descrizione delle y:  +/- e rappresenta se la carta di credito e’ stata approvata o meno\\
\item Descrizione sintetica del problema: Stabilire se un utente è elegibile ad avere una carta di credito o meno\\
\item Tipo di modello applicato:  GDA (tutti i tipi), Logistic Regression (buondary lineare e quadratico) con regolarizzazione L2.\\
\item Tipo di stima (MLE/MAP/etc) oppure metrica (o metriche) usata per valutare le peformance del vostro : Misclassification error\\
\item Descrizione sintetica dei risultati ottenuti: Differenze sostanziali tra i classificatori generativi e discriminativi, non ce ne sono. Una differenza si nota solo in algorirmi che presentano un numero di parametri elevati\\
\item Linguaggio di programmazione utilizzato: MATLAB\\
\item Libreria o package utilizzati : pmtk3
\end{itemize}


\begin{thebibliography}{10}
\bibitem{Dataset}
	John Ross Quinlan \\
	\emph{Credit Approval Data Set} \\
	\url{http://archive.ics.uci.edu/ml/datasets/Credit+Approval}
\bibitem{Features}
	John Ross Quinlan \\
	\emph{Credit Approval Data Set} \\
	\url{ https://archive.ics.uci.edu/ml/machine-learning-databases/credit-screening/crx.names	}
\end{thebibliography}
\end{document}